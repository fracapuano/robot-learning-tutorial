\begin{figure}[h!]
    \centering
    \captionsetup{type=figure}

    \begin{minipage}[t]{0.48\linewidth}
        \centering
        \setlength{\tabcolsep}{6pt}
        \renewcommand{\arraystretch}{1.2}
        \resizebox{0.8\linewidth}{!}{
        \begin{tabular}{cccccc}
            \toprule
            \multirow{2}{*}{\textbf{N}} & \multicolumn{5}{c}{\textbf{Success Rate (\%) — LIBERO}} \\\cmidrule(lr){2-6}
             & \textbf{S} & \textbf{O} & \textbf{G} & \textbf{10} & \textbf{Avg} \\
            \midrule
            8           & 77 & 88 & 86 & 49 & 75.0  \\
            16          & 88 & 91 & 91 & 44 & 78.5  \\
            24          & 86 & 97 & 86 & 49 & 79.5  \\
            32          & 89 & 94 & 85 & 53 & 80.3  \\
            \midrule
            Skip \%2     & 84 & 90 & 83 & 45 & 75.5  \\
            \midrule
            VLM-256M     & 86 & 83 & 75 & 59 & 75.8  \\
            \bottomrule
        \end{tabular}
        }
        \captionof{table}{\textbf{Skipping VLM layers.} Skipping layers from a large VLM (here 500M parameters) yields better results than downsizing the VLM. Skipping every second layer is a competitive baseline.}
        \label{tab:ablation_vlm_layers}
    \end{minipage}
    \hfill
    \begin{minipage}[t]{0.48\linewidth}
        \centering
        \setlength{\tabcolsep}{6pt}
        \renewcommand{\arraystretch}{1.2}
        \resizebox{0.8\linewidth}{!}{
        \begin{tabular}{cccccc}
            \toprule
            \textbf{Expert width} & \multicolumn{5}{c}{\textbf{Success Rate (\%) — LIBERO}} \\\cmidrule(lr){2-6}
             \textbf{(w.r.t. VLM)} & \textbf{S} & \textbf{O} & \textbf{G} & \textbf{10} & \textbf{Avg} \\
            \midrule
            $\times$1.00 & 87 & 96 & 90 & 56 & 82.3 \\
            $\times$0.75 & 82 & 89 & 84 & 55 & 77.5 \\
            $\times$0.50 & 89 & 94 & 85 & 53 & 80.3 \\
            $\times$0.25 & 76 & 97 & 83 & 39 & 73.8 \\
            \bottomrule
        \end{tabular}
        }
        \captionof{table}{\textbf{Expert capacity.} Adjusting the expert's hidden size affects performance. Larger capacities yield better success rates.}
        \label{tab:ablation_expert_capacity}
    \end{minipage}
    
    % \vspace{0.5cm}

    % \begin{minipage}[t]{0.45\linewidth}
    %     \centering
    %     \setlength{\tabcolsep}{6pt}
    %     \renewcommand{\arraystretch}{1.2}
    %     \resizebox{1.0\linewidth}{!}{
    %     \begin{tabular}{lccccc}
    %         \toprule
    %         \textbf{Expert Variant} & \multicolumn{5}{c}{\textbf{Success Rate (\%) — LIBERO}} \\\cmidrule(lr){2-6}
    %          & \textbf{S} & \textbf{O} & \textbf{G} & \textbf{10} & \textbf{Avg} \\
    %         \midrule
    %         Full             & 89 & 94 & 85 & 53 & 80.3  \\
    %         Skip 1 Layer     & 84 & 90 & 83 & 45 & 75.5  \\
    %         VLM-256M         & 86 & 83 & 75 & 59 & 75.8  \\
    %         \bottomrule
    %     \end{tabular}
    %     }
    %     \captionof{table}{\textbf{Shallower action expert.} Using a shallower expert reduces performance compared to the full version, though smaller VLMs with full experts perform similarly.}
    %     \label{tab:ablation_shallow_expert}
    % \end{minipage}
\end{figure}
