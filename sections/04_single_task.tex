\section{Robot (Imitation) Learning}
\label{sec:learning-bc-single}

\epigraph{\emph{The best material model for a cat is another, or preferably the same cat}}{Norbert Wiener}

% Contents
% - Introduce the need for BC considering the limitations of RL
% - Derive the motivation for BC from: 
% - Exploration: It is difficult to learn from sparse
% rewards (unless data is cheap and you don’t care
% about seeing lots of failures).
% - Reward design: Hand-designing rewards aligned
% with human objectives and preferences is hard.
% - Success detection
% - Resets

% Finish arguing for the benefits of learning distributions over state-actions pairs instead of single actions, and that these distributions can be learned using generative modeling.


Learning from human demonstrations provides a pragmatic alternative to the reinforcement-learning pipeline discussed in Section~\ref{sec:learning-rl}.
In real-world robotics, online exploration is typically \highlight{costly and potentially unsafe}, and (dense) reward design is \highlight{brittle and task-specific}.
In general, success detection often may require bespoke instrumentation, and episodic training demands reliable resets---all factors that complicate training RL algorithms on hardware at scale.
Behavioral Cloning (BC) sidesteps these constraints by casting control an imitation learning problem, leveraging previously collected expert demonstrations.
In this way, the objectives, preferences, and success criteria are implicitly encoded in the data, obviating hand-crafted reward shaping and reducing early-stage exploratory failures.

Formally, let \( \mathcal D = \{ \tau^{(i)} \}_{i=1}^N \) be a set of expert trajectories, with \( \tau^{(i)} = \{(o_t^{(i)}, a_t^{(i)})\}_{t=0}^{T_i} \), where \(o_t \in \obsspace \) denotes observations (e.g., images and proprioception), and \(a_t \in \actionspace \) the expert actions.
Figure~\ref{fig:ch4-bc-trajectories} shows observations over sample trajectories for a dataset collected by a human demonstrating a manipulation task on the SO-100 arm, while Figure~\ref{fig:ch4-observation-action-mapping} shows \( (o_t, a_t) \)-pairs for the same dataset, with the actions commanded by the human expert illustrated alongside the corresponding observation.
Note that differently from Section~\ref{sec:learning-rl}, in this context \( \mathcal D \) indicates an offline dataset (and not the dynamics), collecting \( N \) length-\( T_i \) \emph{reward-free} (expert) human trajectories \( \tau^{(i)} \) (and not the a trajectory for a given MDP).
In principle, (expert) trajectories \( \tau^{(i)} \) can have different lengths since demonstrations might exhibit multi-modal strategies to attain the same goal, resulting in possibly multiple, different behaviors.

Behavioral Cloning (BC)~\citep{pomerleauALVINNAutonomousLand1988a} aims at synthetizing these behaviors learning the mapping from observations to actions, and in its simplest formulation can be effectively be tackled as a \emph{supevised} learning problem, where one learns a mapping \( a_t = \mu(o_t) \) by solving
\begin{equation}\label{eq:loss-minimization-SL}
    \min_{\mu} \mathbb{E}_{(o_t, a_t) \sim \mathcal{D}^*} \mathcal L(a_t, \mu(o_t)),
\end{equation}
for a given risk function \( \mathcal L:  \mathcal A \times \mathcal A \mapsto \mathbb{R}, \ \mathcal L (a, a^\prime) \).
The expert's joint observation-action distribution \( \mathcal D^* \) is typically assumed to be unknown, similarily to a typical Supervised Learning (SL) framework~\citep{shalev-shwartzUnderstandingMachineLearning2014}.
However, differently from standard SL's assumptions, the samples collected in \( \mathcal D \) are \emph{not} typically i.i.d., as expert demonstrations are collected \emph{sequentially} in trajectories.
In practice, this aspect can be partially mitigated by considering pairs in a non-sequential order---\emph{shuffling} the samples in \(\mathcal D \)---so that the expected risk under \( \mathcal D^* \) can be estimated using MC estimates, drawing \((o_t, a_t)\) from \( \mathcal D \), although estimates can in general be less accurate.

The BC formulation affords several operational advantages in robotics.
First, training happens offline and typically uses expert human demonstration data, hereby severily limiting risks.
Second, reward design is unnecessary in BC, as demonstrations already reflect human intent and task completion, mitigating misalignment and the risk of specification gaming inherent in RL.
Third, because expert trajectories encode terminal conditions, success detection and resets are implicit in the dataset.
Finally, BC scales naturally with growing corpora of demonstrations collected across tasks, embodiments, and environments.

However, point-estimate policies \( \mu : \obsspace \mapsto \actionspace \) learned by solving \ref{eq:loss-minimization-SL} have been observed to suffer from (1) compounding errors and (2) poor fit to multimodal distributions~\citep{florenceImplicitBehavioralCloning2022, keGraspingChopsticksCombating2020} (Figure~\ref{fig:ch4-issues-with-bc}).
Besides sequentiality in \( \mathcal D \), compounding errors are due to \emph{covariate shift}, whereby even \( \epsilon \)-prediction errors \( 0 < \Vert \mu(o_t) - a_t \Vert \leq \epsilon \) can quickly drive the policy into out-of-distribution states, incuring in a higher risk value thus quickly compounding errors (Figure~\ref{fig:ch4-issues-with-bc}, left).
Further, point-estimate policies fail to learn \emph{multimodal} (in the stastical sense) targets, common in robotics problems where multiple trajectories can be equally as good towards the realization of a given objective (e.g.,  symmetric grasps, Figure~\ref{fig:ch4-issues-with-bc}, left). 
In particular, unimodal regressors tend to average across modes, yielding indecisive or even unsafe commands~\citep{florenceImplicitBehavioralCloning2022}.

Addressing both limitations,~\citet{florenceImplicitBehavioralCloning2022} introduced learning the generative model underlying samples \( \mathcal D \).
Instead of learning to predict point estimates \( \hat a_t = \mu(o_t) \),~\citet{florenceImplicitBehavioralCloning2022}  model \( p_\theta(a_t \mid \text{context}_t) \).
Distributional policies have been found to flexibly learn from human-collected demonstrations, capturing the inherent ambiguity of learning multimodal distributions.


\subsection{A (Concise) Introduction to Generative Modeling}

% Generative Modeling
What is generative modeling
What are its premises

% VAEs
What is the intuition behind VAEs
A derivation of the VAE loss
Reconstruction and Regularization, why

% Diffusion
What is the intuition behind diffusion
Deriving the diffusion loss

% Flow Matching
What is the intuition behind FM


\subsection{Action Chunking with Transformers}
Model Architecture
Training Procedure

\subsubsection{Code Example: Learning ACT}
\todo{using act example}

\subsection{Diffusion Policy}
fill later

\subsubsection{Code Example: Learning Diffusion Policies}
\todo{using diffusion policy example}

\subsection{Optimized Inference via Decoupling Action Prediction and Execution}
Implicit World Modeling 
Async inference tightens control loop
Analyzing async inference

\subsubsection{Code Example: Deploy Optimized Policies}
\todo{async inference example}